\documentclass[12pt]{article}
\usepackage[serbian]{babel}
\usepackage {fontspec, amsmath, amssymb}
\defaultfontfeatures{Ligatures=TeX}

\addto\captionsserbian{%
  \renewcommand{\abstractname}%
    {Abstrakt}%
}

\title{Simulacija te\v cnosti u dve dimenzije}
\author{
        Daniel Sila\dj i \\
        Gimnazija "Jovan Jovanovi\'{c} Zmaj"\\
		Novi Sad
}
\date{\today}

\begin{document}
\maketitle

\begin{abstract}
This is the paper's abstract \ldots
\end{abstract}

\section{Uvod}\label{uvod}
    Razni fluidi ( ) su oduvek bili sastavni deo na\v sih \v zivota, pa je pored sveop\v steg razvoja tehnologije bilo pitanje vremena kad \'ce \v covek po\v zeleti da predvidi njihovo kretanje, odnosno -- da ih simulira.
    Prirodno, i simulacije su postale deo na\v sih \v zivota, i to, sa jedne strane u vidu vremenske prognoze, aerodinami\v cki oblikovanih automobila, aviona i raketa, a sa druge strane kao specijalni efekti u filmovima i kompjuterskim igrama.

    \subsection{Istorija}\label{istorija}
        Prve korake u ovoj oblasti su napravili Claude-Louis Navier i George Gabriel Stokes 1822, postaviv\v si Navier-Stoksove jedna\v cine.
        One \v cine osnovu mnogih modela atmosfere, okeana, vodovoda, krvotoka, a koriste se i u ispitivanju aerodinami\v cnosti aviona i automobila.
        Ipak, ove jedna\v cine imaju jedan veliki nedostatak: ne zna se da li imaju re\v senje za proizvoljno stanje fluida u 3 dimenzije.
        \v Stavi\v se, to pitanje predstavlja jedan od sedam milenijumskih problema Clayovog instituta za matematiku.
        Zbog toga se i danas radi na pronala\v zenju \v sto br\v zih (za izra\v cunavanje), ili \v sto preciznijih aproksimacija ovih jedna\v cina, koje nam garantuju da \'ce re\v senje postojati.

        Prve kompjuterske simulacije su bile delo stru\v cnjaka iz NASA-e, ustanove koja je u tom trenutku jedina imala kompjutere dovoljno jake da simuliraju vazduh u realnom vremenu, pa makar to bilo u dve dimenzije. Bitan pomak u re\v savanju Navier-Stokesovih jedna\v cina u 3D na\v cinjen je u \cite{Foster:1996:RAL:244304.244315}, oslanjaju\'ci se na \cite{harlow:2182}.

        Osnovu ovog rada \v cini metod Smoothed Particle Hydrodynamics (SPH), otkriven 1977, nezavisno u \cite{1977MNRAS.181..375G} i \cite{1977AJ.....82.1013L}. U oba rada SPH je iskori\v s\'cen za modeliranje zvezda, i zbog toga nije pravljen da radi u realnom vremenu. Tek u \cite{Muller:2003:PFS:846276.846298} je dat pojednostavljeni algoritam, pogodan za izvr\v savanje u realnom vremenu.

    \subsection{Motivacija i ostala veselja}\label{motivacija}
        blabla 

\section{Kori\v s\'cene metode}
    

\bibliographystyle{abbrv}
\bibliography{2DSPH}

\end{document}
This is never printed
