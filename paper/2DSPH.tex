\documentclass[12pt]{article}
\usepackage[serbian]{babel}
\usepackage {fontspec, amsmath, amssymb}
\defaultfontfeatures{Ligatures=TeX}

\addto\captionsserbian{%
  \renewcommand{\abstractname}%
    {Abstrakt}%
}

\title{Simulacija te\v cnosti u dve dimenzije}
\author{
        Daniel Sila\dj i \\
        Gimnazija "Jovan Jovanovi\'{c} Zmaj"\\
		Novi Sad
}
\date{\today}

\begin{document}
\maketitle

\begin{abstract}
This is the paper's abstract \ldots
\end{abstract}

\section{Uvod}
Razni fluidi ( ) su oduvek bili sastavni deo na\v sih \v zivota, pa je pored sveop\v steg razvoja tehnologije bilo pitanje vremena kad \'ce \v covek po\v zeleti da predvidi njihovo kretanje, odnosno -- da ih simulira.
Prirodno, i simulacije su postale deo na\v sih \v zivota, i to, sa jedne strane u vidu vremenske prognoze, aerodinami\v cki oblikovanih automobila, aviona i raketa, a sa druge strane kao specijalni efekti u filmovima i kompjuterskim igrama.

\subsection{Istorija}
Prve korake u ovoj oblasti su napravili Claude-Louis Navier i George Gabriel Stokes 1822, postaviv\v si Navier-Stoksove jedna\v cine.
One \v cine osnovu mnogih modela atmosfere, okeana, vodovoda, krvotoka, a koriste se i u ispitivanju aerodinami\v cnosti aviona i automobila.
Ipak, ove jedna\v cine imaju jedan veliki nedostatak: ne zna se da li imaju re\v senje za proizvoljno stanje fluida u 3 dimenzije. 
\v Stavi\v se, to pitanje predstavlja jedan od sedam milenijumskih problema Clayovog instituta za matematiku. 
Zbog toga se i danas radi na pronala\v zenju \v sto br\v zih (za izra\v cunavanje), ili \v sto preciznijih aproksimacija ovih jedna\v cina, 
koje nam garantuju da \'ce re\v senje postojati.




\section{Previous work}\label{previous work}
A much longer \LaTeXe{} example was written by Gil~\cite{Gil:02}.

\section{Results}\label{results}
In this section we describe the results.

\section{Conclusions}\label{conclusions}
We worked hard, and achieved very little.

\bibliographystyle{abbrv}
\bibliography{main}

\end{document}
This is never printed
